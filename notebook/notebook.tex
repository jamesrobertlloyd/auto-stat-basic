\documentclass[twoside,11pt]{article}

% Any additional packages needed should be included after jmlr2e.
% Note that jmlr2e.sty includes epsfig, amssymb, natbib and graphicx,
% and defines many common macros, such as 'proof' and 'example'.
%
% It also sets the bibliographystyle to plainnat; for more information on
% natbib citation styles, see the natbib documentation, a copy of which
% is archived at http://www.jmlr.org/format/natbib.pdf

\usepackage{jmlr2e}

\usepackage{listings}
%\usepackage{algorithm}
%\usepackage{algorithmic}
\usepackage{amssymb,amsmath}
%\usepackage{graphicx}
\usepackage{preamble}
%\usepackage{natbib}
%%%% REMEMBER ME!
%\usepackage[draft]{hyperref}
\usepackage{hyperref}
\usepackage{color}
\usepackage{url}
%\usepackage{wasysym}
%\usepackage{subfigure}
%\usepackage{tabularx}
\usepackage{booktabs}
%\usepackage{bm}
%\newcommand{\theHalgorithm}{\arabic{algorithm}}
\definecolor{mydarkblue}{rgb}{0,0.08,0.45}
\hypersetup{ %
    pdftitle={},
    pdfauthor={},
    pdfsubject={},
    pdfkeywords={},
    pdfborder=0 0 0,
    pdfpagemode=UseNone,
    colorlinks=true,
    linkcolor=mydarkblue,
    citecolor=mydarkblue,
    filecolor=mydarkblue,
    urlcolor=mydarkblue,
    pdfview=FitH}

\setlength{\marginparwidth}{0.6in}
\input{include/commenting.tex}

%% For submission, make all render blank.
%\renewcommand{\LATER}[1]{}
%\renewcommand{\fLATER}[1]{}
%\renewcommand{\TBD}[1]{}
%\renewcommand{\fTBD}[1]{}
%\renewcommand{\PROBLEM}[1]{}
%\renewcommand{\fPROBLEM}[1]{}
%\renewcommand{\NA}[1]{#1}  %% Note, NA's pass through!

% Definitions of handy macros can go here

% Heading arguments are {volume}{year}{pages}{submitted}{published}{author-full-names}

%\jmlrheading{Volume}{Year}{Pages}{Submitted}{Published}{James Robert Lloyd}

% Short headings should be running head and authors last names

\ShortHeadings{Auto stat basic architecture}{Lloyd et alia}
\firstpageno{1}

\begin{document}

\lstset{language=Lisp,basicstyle=\ttfamily\footnotesize} 

\title{Designing an API for the automatic statistician}

\author{\name James Robert Lloyd \email jrl44@cam.ac.uk \\
       \addr 
       Machine Learning Group \\
       Department of Engineering\\
       University of Cambridge\\
       \AND
       \name Others\dots}

\editor{Editor}

\maketitle

\begin{abstract}
Abstract
\end{abstract}

%\begin{keywords}
%  Gaussian processes
%\end{keywords}

\section{Introduction}

Intro

\section{Simplifying assumptions}

We will initially make many simplifying assumptions with a view to relaxing them as we expand the basic architecture:

\begin{itemize}
  \item Data assumed to be an exchangeable sequence of vectors
  \item Only building conditional models of a single output
  \item Minimal user interaction
  \item \dots
\end{itemize}

\section{Examples of desired functionality}

This describes tasks the system should be able to perform.
Some are within the scope of the current simplifying assumptions, others will require additional interfaces / messaging protocols.

\subsection{Cross validated model selection}

\subsection{Bayesian model selection}

\subsection{Bayesian model averaging}

\subsection{Bayesian optimisation of model selection}

\subsection{Cross vaidated ensemble construction}

\section{Other desiderata}

\begin{itemize}
  \item Concurrency
  \item Language independence
  \item Scalability
  \item \dots
\end{itemize}

\section{Definition of architecture}

Concurrency suggests a framework involving actors passing messages to each other.
The messages should be language independent, suggesting perhaps an XML syntax of messages?

\subsection{Overview}

\subsection{Minimum messages}

Things like start, pause, clear, load.

\subsection{Optional messages}

Things like returning models or cross validated errors.

\subsection{Future messages}

Types of message that might be useful in future versions of the system. 

\newpage

%\appendix
%\section*{Appendix A.}
%Appendix

\vskip 0.2in
\bibliography{library}

\end{document}
